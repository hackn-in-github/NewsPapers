%#!ptex2pdf -l -u -ot '-synctex=1' 新聞配達部数
%#!lualatex -synctex=1 新聞配達部数
%#!lualatex --jobname=前期/新聞配達部数-fig1 新聞配達部数
%#!xelatex -synctex=1 新聞配達部数
%#!xelatex --jobname=前期/新聞配達部数-fig1 新聞配達部数
%#!uplatex -shell-escape 新聞配達部数
%#!uplatex 新聞配達部数
%#LPR dvipdfmx -p a4 -r 1200 新聞配達部数
% #LPR dvips -Ppdf -f 新聞配達部数.dvi | lpr
%#MAKEINDEX mendex -g -s mystyle.ist 新聞配達部数
%#!platex -shell-escape 新聞配達部数
%#!platex 新聞配達部数
%%% sudo tlmgr update --self --all
%\documentclass[a4j,xelatex,ja=standard,magstyle=nomag*]{bxjsreport}% ドライバ指定は無し
%\XeTeXgenerateactualtext=1 %PDF ファイルの検索やコピー & ペーストがうまく動作しない場合指定
%\RequirePackage{luatex85}%\pdfナントカでエラーになったときに必要ってか02/01/2017時点で必要
%hyperrefを新しい物にして不要になる
\documentclass[a4j,papersize,landscape,uplatex,dvipdfmx,ja=standard,magstyle=nomag*]{bxjsreport}% ドライバ指定は無し
%\documentclass[a4j,landscape,lualatex,ja=standard,magstyle=nomag*]{bxjsreport}% ドライバ指定は無し
\setpagelayout{margin=15mm}%余白を若干狭く
\usepackage{metalogo}% \XeLaTeX ロゴのため
%% XeLaTeX
%\usepackage{zxjatype}
%\setjamainfont[BoldFont=IPAexGothic]{IPAexMincho}
%\setjasansfont{IPAexGothic}
%\setmonofont{IPAexGothic}
%\setjamainfont[BoldFont=MogaMinExB]{MogaMinEx}
%\setjasansfont{MogaGoEx}
%\setmonofont{MogaGo}
%\setjamainfont[BoldFont=YOzNb]{YOzN}
%\setjasansfont{YOzNb}
%\setmonofont{YOzNF}
%\setjamainfont[BoldFont=Ume P Gothic]{Ume P Mincho}
%\setjasansfont{Ume P Gothic}
%\setmonofont{Ume Gothic}
%\setjamainfont[BoldFont=TakaoPGothic]{TakaoPMincho}
%\setjasansfont{TakaoPGothic}
%\setmonofont{TakaoGothic}
%\setjamainfont[BoldFont=mikachan-PB]{mikachan-P}
%\setjasansfont{mikachan-PB}
%\setmonofont{mikachan}
%\setjamainfont{Migu 1P}
%\setjasansfont{Migu 1P}
%\setmonofont{Migu 1M}
%\setjamainfont{M+ 1p}
%\setjasansfont{M+ 1p}
%\setmonofont{M+ 1m}
%% LuaLaTeX-ja こっちの方が綺麗かも…
%\setmainjfont[BoldFont=IPAexGothic]{IPAexMincho}
%\setsansjfont{IPAexGothic}
%\setmonofont{IPAexGothic}
%\setmainjfont[BoldFont=MogaMinExB]{MogaMinEx}
%\setsansjfont{MogaGoEx}
%\setmonofont{MogaGo}
%\setmainjfont[BoldFont=YOzNb]{YOzN}
%\setsansjfont{YOzNb}
%\setmonofont{YOzNF}
%\setmainjfont[BoldFont=Ume P Gothic]{Ume P Mincho}
%\setsansjfont{Ume P Gothic}
%\setmonofont{Ume Gothic}
%\setmainjfont[BoldFont=TakaoPGothic]{TakaoPMincho}
%\setsansjfont{TakaoPGothic}
%\setmonofont{TakaoGothic}
%\setmainjfont[BoldFont=mikachan-PB]{mikachan-P}
%\setsansjfont{mikachan-PB}
%\setmonofont{mikachan}
%\setmainjfont{Migu 1P}
%\setsansjfont{Migu 1P}
%\setmonofont{Migu 1M}
%\setmainjfont{M+ 1p}
%\setsansjfont{M+ 1p}
%\setmonofont{M+ 1m}
\makeatletter
\newif\ifpapersize
\newif\if@jsclass
\newif\iftombow
%% \ruby対策 %%
\@ifundefined{kanjiskip}{\newlength\kanjiskip}{}%
\@ifundefined{xkanjiskip}{\newlength\xkanjiskip}{}%
%%%%%%%%%%%%%%
\@ifundefined{zh}{\newlength\zh}{}%
\makeatother
\setlength{\zh}{1\zw}
\usepackage{emath,EMproof,emathMw}
%% edaenumerate環境の上部の間隔設定 edatopsep=-1\zh
%---- mathabx使用時に指定
\let\therefore\relax
\let\because\relax
\usepackage{mathabx}%
%---------------------
%\usepackage{newtxmath,newtxtext}%
%\usepackage{mathptmx}%newtxの方がいいかな?
\let\EMforeach=\foreach
\MWsep{4pt}%
\usepackage[enumitem]{tikzxelatexsetup}
\usepackage{hyperref}
\hypersetup{%
 bookmarkstype=toc,%
 bookmarksnumbered=true,%
 colorlinks=true,%
 urlcolor=blue,%
 linkcolor=blue,%
 setpagesize=false,%
 pdftitle={},%
 pdfauthor={飯島 徹},%
 pdfsubject={},%
 pdfkeywords={2016年,平成28年,前期,二次試験,数学,問題,詳解}%
}%
%\title{{\XeLaTeX}で日本語文書}
%\author{七篠\ 権兵衛}
% \today の出力はJSクラスと同様に西暦になる
%(1)二行ならべる
%(2)ベースライン間の距離が\baselineskipになるようにする
%(3)上の行の「下端」と下の行の「上端」の間の縦方向の距離を測る
%(4)(3)で測定した値と\lineskiplimitを比較する
%(4-1)(3)の値>\lineskipならばそのまま
%(4-2)そうでなければ,上の行の「下端」と下の行の「上端」の間を\lineskipにする
\pagestyle{empty}%
\lineskiplimit=1pt\relax
\lineskip=2pt\relax
\pgfrealjobname{新聞配達部数}%
\SetPath{前期/}%
%\includeonly{}%
\begin{document}
\hfill\today

\tikzset{FN/.style={font=\normalsize}}%
\begin{tikzpicture}[font=\Huge,yscale=1,>={Stealth[length=4mm]}]
 \SetPoint*{O1(26,0);O2($(O1)+(-.5,0)$);O3($(O2)+(-.5,0)$);O4($(O3)+(-.5,0)$);O5($(O4)+(-.5,0)$);O6($(O5)+(-.5,0)$)}%
 \node(仲町12)at(0,0)[e]{仲町1・2丁目[読売48部(金・土・日読売$+1$)]};
 \node(仲町3)at(0,-1)[e]{仲町3丁目[読売15部(土・日報知$+1$)]};
 \node(東町)at(0,-2)[e]{東町[読売19部]};
 \node(柏台)at(0,-3)[e]{柏台[読売6部]};
 \node(本町1)at(0,-4)[e]{本町1丁目[読売15部・報知1部(土・日・月読売$+1$)]};
 \node(本町2)at(0,-5)[e]{本町2丁目[読売24部・報知1部]};
 %
 \node(後仲町2)at(0,-7)[e]{組換後:仲町2丁目[読売17部]};
 \node(後仲町3)at(0,-8)[e]{組換後:仲町3丁目[読売16部(土・日報知$+1$)]};
 \node(後東町)at(0,-9)[e]{組換後:東町[読売23部]};
 \node(後仲町12)at(0,-10)[e]{組換後:仲町1・2丁目[読売32部(金・土・日読売$+1$)]};
 \node(後本町1)at(0,-11)[e]{組換後:本町1丁目[読売15部・報知1部(土・日・月読売$+1$)]};
 \node(後本町2)at(0,-12)[e]{組換後:本町2丁目[読売24部・報知1部]};
 %
 \draw[->,very thick](仲町12.east)--(仲町12-|O1)--(後仲町12-|O1)--(後仲町12.east)node[FN,pos=.5,fill=white]{読32$+1$};
 \draw[->,very thick]($(後仲町2-|O1)+(0,.2)$)--($(後仲町2.east)+(0,.2)$)node[FN,pos=.5,fill=white]{読16};
 \draw[->](仲町3.east)--(仲町3-|O2)--($(後仲町3-|O2)+(0,.2)$)--($(後仲町3.east)+(0,.2)$)node[pos=.5,FN,fill=white]{読15報$+1$};
 \draw[->,dotted,very thick]($(後仲町3-|O4)+(0,-.2)$)--($(後仲町3.east)+(0,-.2)$)node[pos=.5,FN,fill=white]{読1};
 \draw[->,dash dot,very thick](東町.east)--(東町-|O3)--($(後東町-|O3)+(0,.2)$)--($(後東町.east)+(0,.2)$)node[pos=.5,FN,fill=white]{読19};
 \draw[->,dotted,very thick]($(後仲町2-|O4)+(0,-.2)$)--($(後仲町2.east)+(0,-.2)$)node[FN,pos=.5,fill=white]{読1};
 \draw[->,dotted,very thick](柏台.east)--(柏台-|O4)--($(後東町-|O4)+(0,-.2)$)--($(後東町.east)+(0,-.2)$)node[pos=.5,FN,fill=white]{読4};
 \draw[->,thick](本町1.east)--(本町1-|O5)--(後本町1-|O5)node[e,FN]{読$15+1$報1}--(後本町1.east);
 \draw[->,thick,dash dot dot](本町2.east)--(本町2-|O6)--(後本町2-|O6)--(後本町2.east)node[pos=.5,FN,fill=white]{読24報1};
\end{tikzpicture}
\end{document}